

\chapter{午後の授業}

「ではみなさんは、そういうふうに川だと\ruby{言}{い}われたり、\ruby{乳}{ちち}の\ruby{流}{なが}れたあとだと\ruby{言}{い}われたりしていた、このぼんやりと白いものがほんとうは何かご\ruby{承知}{しょう|ち}ですか」先生は、\ruby{黒板}{こく|ばん}につるした大きな黒い\ruby{星座}{せい|ざ}の図の、上から下へ白くけぶった\ruby{銀河帯}{ぎん|が|たい}のようなところを\ruby{指}{さ}しながら、みんなに\ruby{問}{と}いをかけました。

カムパネルラが手をあげました。それから四、五人手をあげました。ジョバンニも手をあげようとして、\ruby{急}{いそ}いでそのままやめました。たしかにあれがみんな星だと、いつか\ruby{雑誌}{ざっ|し}で読んだのでしたが、このごろはジョバンニはまるで毎日教室でもねむく、本を読むひまも読む本もないので、なんだかどんなこともよくわからないという\ruby{気持}{き|も}ちがするのでした。

ところが先生は早くもそれを見つけたのでした。

「ジョバンニさん。あなたはわかっているのでしょう」

ジョバンニは\ruby{勢}{いきお}いよく立ちあがりましたが、立ってみるともうはっきりとそれを答えることができないのでした。ザネリが前の\ruby{席}{せき}からふりかえって、ジョバンニを見てくすっとわらいました。ジョバンニはもうどぎまぎしてまっ赤になってしまいました。先生がまた\ruby{言}{い}いました。

「大きな\ruby{望遠鏡}{ぼう|えん|きょう}で\ruby{銀河}{ぎん|が}をよっく\ruby{調}{しら}べると\ruby{銀河}{ぎん|が}はだいたい何でしょう」

やっぱり星だとジョバンニは思いましたが、こんどもすぐに答えることができませんでした。

先生はしばらく\ruby{困}{こま}ったようすでしたが、\ruby{眼}{め}をカムパネルラの方へ\ruby{向}{む}けて、

「ではカムパネルラさん」と\ruby{名指}{な|ざ}しました。

するとあんなに元気に手をあげたカムパネルラが、やはりもじもじ立ち上がったままやはり答えができませんでした。

先生は\ruby{意外}{い|がい}なようにしばらくじっとカムパネルラを見ていましたが、\ruby{急}{いそ}いで、

「では、よし」と\ruby{言}{い}いながら、自分で星図を\ruby{指}{さ}しました。

「このぼんやりと白い\ruby{銀河}{ぎん|が}を大きないい\ruby{望遠鏡}{ぼう|えん|きょう}で見ますと、もうたくさんの小さな星に見えるのです。ジョバンニさんそうでしょう」

ジョバンニはまっ\ruby{赤}{か}になってうなずきました。けれどもいつかジョバンニの\ruby{眼}{め}のなかには\ruby{涙}{なみだ}がいっぱいになりました。そうだ\ruby{僕}{ぼく}は知っていたのだ、もちろんカムパネルラも知っている、それはいつかカムパネルラのお父さんの\ruby[g]{博士}{はかせ}のうちでカムパネルラといっしょに読んだ\ruby{雑誌}{ざっ|し}のなかにあったのだ。それどこでなくカムパネルラは、その\ruby{雑誌}{ざっ|し}を読むと、すぐお父さんの\ruby{書斎}{しょ|さい}から\ruby{巨}{おお}きな本をもってきて、ぎんがというところをひろげ、まっ黒な\ruby{頁}{ページ}いっぱいに白に\ruby{点々}{てん|てん}のある\ruby{美}{うつく}しい\ruby{写真}{しゃ|しん}を二人でいつまでも見たのでした。それをカムパネルラが\ruby{忘}{わす}れるはずもなかったのに、すぐに\ruby{返事}{へん|じ}をしなかったのは、このごろぼくが、朝にも午後にも\ruby{仕事}{し|ごと}がつらく、学校に出てももうみんなともはきはき\ruby{遊}{あそ}ばず、カムパネルラともあんまり物を\ruby{言}{い}わないようになったので、カムパネルラがそれを知ってきのどくがってわざと\ruby{返事}{へん|じ}をしなかったのだ、そう考えるとたまらないほど、じぶんもカムパネルラもあわれなような気がするのでした。

先生はまた\ruby{言}{い}いました。

「ですからもしもこの天の川がほんとうに川だと考えるなら、その一つ一つの小さな星はみんなその川のそこの\ruby{砂}{すな}や\ruby{砂利}{じゃ|り}の\ruby{粒}{つぶ}にもあたるわけです。またこれを\ruby{巨}{おお}きな\ruby{乳}{ちち}の\ruby{流}{なが}れと考えるなら、もっと天の川とよく\ruby{似}{に}ています。つまりその星はみな、\ruby{乳}{ちち}のなかにまるで\ruby{細}{こま}かにうかんでいる\ruby[g]{脂油}{あぶら}の\ruby{球}{たま}にもあたるのです。そんなら何がその川の水にあたるかと\ruby{言}{い}いますと、それは\ruby{真空}{しん|くう}という光をある\ruby{速}{はや}さで\ruby{伝}{つた}えるもので、\ruby{太陽}{たい|よう}や\ruby{地球}{ち|きゅう}もやっぱりそのなかに\ruby{浮}{う}かんでいるのです。つまりは\ruby{私}{わたし}どもも天の川の水のなかに\ruby{棲}{す}んでいるわけです。そしてその天の川の水のなかから四方を見ると、ちょうど水が深いほど青く見えるように、天の川の\ruby{底}{そこ}の\ruby{深}{ふか}く遠いところほど星がたくさん集まって見え、したがって白くぼんやり見えるのです。この\ruby{模型}{も|けい}をごらんなさい」

先生は中にたくさん光る\ruby{砂}{すな}のつぶのはいった大きな\ruby{両面}{りょう|めん}の\ruby{凸}{とつ}レンズを\ruby{指}{さ}しました。

「天の川の形はちょうどこんななのです。このいちいちの光るつぶがみんな\ruby{私}{わたし}どもの\ruby{太陽}{たい|よう}と同じようにじぶんで光っている星だと考えます。私どもの\ruby{太陽}{たい|よう}がこのほぼ中ごろにあって\ruby{地球}{ち|きゅう}がそのすぐ近くにあるとします。みなさんは夜にこのまん中に立ってこのレンズの中を見まわすとしてごらんなさい。こっちの方はレンズが\ruby{薄}{うす}いのでわずかの光る\ruby{粒}{つぶ}すなわち星しか見えないでしょう。こっちやこっちの方はガラスが\ruby{厚}{あつ}いので、光る\ruby{粒}{つぶ}すなわち星がたくさん見えその遠いのはぼうっと白く見えるという、これがつまり今日の\ruby{銀河}{ぎん|が}の\ruby{説}{せつ}なのです。そんならこのレンズの大きさがどれくらいあるか、またその中のさまざまの星についてはもう時間ですから、この\ruby{次}{つぎ}の理科の時間にお話します。では今日はその\ruby{銀河}{ぎん|が}のお\ruby{祭}{まつ}りなのですから、みなさんは外へでてよくそらをごらんなさい。ではここまでです。本やノートをおしまいなさい」

そして教室じゅうはしばらく\ruby{机}{つくえ}の\ruby{蓋}{ふた}をあけたりしめたり本を\ruby{重}{かさ}ねたりする音がいっぱいでしたが、まもなくみんなはきちんと立って\ruby{礼}{れい}をすると教室を出ました。

\chapter[活版所]{\ruby{活版所}{かっ|ぱん|じょ}}

ジョバンニが学校の門を出るとき、同じ組の七、八人は家へ帰らずカムパネルラをまん中にして\ruby{校庭}{こう|てい}の\ruby{隅}{すみ}の\ruby{桜}{さくら}の木のところに\ruby{集}{あつ}まっていました。それはこんやの\ruby{星祭}{ほし|まつ}りに青いあかりをこしらえて川へ\ruby{流}{なが}す\ruby{烏瓜}{からす|うり}を\ruby{取}{と}りに行く\ruby{相談}{そう|だん}らしかったのです。

けれどもジョバンニは手を大きく\ruby{振}{ふ}ってどしどし学校の\ruby{門}{もん}を出て来ました。すると町の家々ではこんやの\ruby{銀河}{ぎん|が}の\ruby{祭}{まつ}りにいちいの\ruby{葉}{は}の\ruby{玉}{たま}をつるしたり、ひのきの\ruby{枝}{えだ}にあかりをつけたり、いろいろしたくをしているのでした。

家へは帰らずジョバンニが町を三つ\ruby{曲}{ま}がってある大きな\ruby{活版所}{かっ|ぱん|じょ}にはいって\ruby{靴}{くつ}をぬいで上がりますと、\ruby{突}{つ}き当たりの大きな\ruby{扉}{とびら}をあけました。中にはまだ\ruby{昼}{ひる}なのに\ruby{電燈}{でん|とう}がついて、たくさんの\ruby{輪転機}{りん|てん|き}がばたりばたりとまわり、きれで頭をしばったりラムプシェードをかけたりした人たちが、何か歌うように読んだり数えたりしながらたくさん\ruby{働}{はたら}いておりました。

ジョバンニはすぐ入口から三番目の高い\ruby[g]{卓子}{テーブル}にすわった人の\ruby{所}{ところ}へ行っておじぎをしました。その人はしばらく\ruby{棚}{たな}をさがしてから、

「これだけ\ruby{拾}{ひろ}って行けるかね」と\ruby{言}{い}いながら、一枚の紙切れを\ruby{渡}{わた}しました。ジョバンニはその人の\ruby[g]{卓子}{テーブル}の足もとから一つの小さな\ruby{平}{ひら}たい\ruby{函}{はこ}をとりだして\ruby{向}{む}こうの\ruby{電燈}{でん|とう}のたくさんついた、たてかけてある\ruby{壁}{かべ}の\ruby{隅}{すみ}の\ruby{所}{ところ}へしゃがみ\ruby{込}{こ}むと、小さなピンセットでまるで\ruby{粟粒}{あわ|つぶ}ぐらいの\ruby{活字}{かつ|じ}を\ruby{次}{つぎ}から\ruby{次}{つぎ}へと\ruby{拾}{ひろ}いはじめました。青い\ruby{胸}{むね}あてをした人がジョバンニのうしろを通りながら、

「よう、虫めがね\ruby{君}{くん}、お早う」と\ruby{言}{い}いますと、近くの四、五人の人たちが声もたてずこっちも\ruby{向}{む}かずに\ruby{冷}{つめ}たくわらいました。

ジョバンニは何べんも\ruby{眼}{め}をぬぐいながら\ruby{活字}{かつ|じ}をだんだんひろいました。

六時がうってしばらくたったころ、ジョバンニは\ruby{拾}{ひろ}った\ruby{活字}{かつ|じ}をいっぱいに入れた\ruby{平}{ひら}たい\ruby{箱}{はこ}をもういちど手にもった紙きれと引き合わせてから、さっきの\ruby[g]{卓子}{テーブル}の人へ\ruby{持}{も}って来ました。その人は\ruby{黙}{だま}ってそれを\ruby{受}{う}け\ruby{取}{と}ってかすかにうなずきました。

ジョバンニはおじぎをすると\ruby{扉}{とびら}をあけて計算台のところに来ました。すると\ruby{白服}{しろ|ふく}を\ruby{着}{き}た人がやっぱりだまって小さな\ruby{銀貨}{ぎん|か}を一つジョバンニに\ruby{渡}{わた}しました。ジョバンニはにわかに顔いろがよくなって\ruby{威勢}{い|せい}よくおじぎをすると、台の下に\ruby{置}{お}いた\ruby{鞄}{かばん}をもっておもてへ\ruby{飛}{と}びだしました。それから元気よく\ruby{口笛}{くち|ぶえ}を\ruby{吹}{ふ}きながらパン\ruby{屋}{や}へ\ruby{寄}{よ}ってパンの\ruby{塊}{かたまり}を一つと\ruby{角砂糖}{かく|ざ|とう}を一\ruby{袋}{ふくろ}買いますといちもくさんに走りだしました。

\chapter{家}

ジョバンニが\ruby{勢}{いきお}いよく帰って来たのは、ある\ruby{裏町}{うら|まち}の小さな家でした。その三つならんだ入口のいちばん\ruby{左側}{ひだり|がわ}には\ruby{空箱}{あき|ばこ}に\ruby{紫}{むらさき}いろのケールやアスパラガスが\ruby{植}{う}えてあって小さな二つの\ruby{窓}{まど}には\ruby{日覆}{ひ|おお}いがおりたままになっていました。

「お母さん、いま帰ったよ。ぐあい\ruby{悪}{わる}くなかったの」ジョバンニは\ruby{靴}{くつ}をぬぎながら言いました。

「ああ、ジョバンニ、お\ruby{仕事}{し|ごと}がひどかったろう。\ruby[g]{今日}{きょう}は\ruby{涼}{すず}しくてね。わたしはずうっとぐあいがいいよ」

ジョバンニは\ruby{玄関}{げん|かん}を上がって行きますとジョバンニのお母さんがすぐ入口の\ruby{室}{へや}に白い\ruby{巾}{きれ}をかぶって\ruby{寝}{やす}んでいたのでした。ジョバンニは\ruby{窓}{まど}をあけました。

「お母さん、今日は\ruby{角砂糖}{かく|ざ|とう}を買ってきたよ。\ruby{牛乳}{ぎゅう|にゅう}に入れてあげようと思って」

「ああ、お前さきにおあがり。あたしはまだほしくないんだから」

「お母さん。\ruby{姉}{ねえ}さんはいつ帰ったの」

「ああ、三時ころ帰ったよ。みんなそこらをしてくれてね」

「お母さんの\ruby{牛乳}{ぎゅう|にゅう}は来ていないんだろうか」

「来なかったろうかねえ」

「ぼく行ってとって来よう」

「ああ、あたしはゆっくりでいいんだからお前さきにおあがり、\ruby{姉}{ねえ}さんがね、トマトで何かこしらえてそこへ\ruby{置}{お}いて行ったよ」

「ではぼくたべよう」

ジョバンニは\ruby{窓}{まど}のところからトマトの\ruby{皿}{さら}をとってパンといっしょにしばらくむしゃむしゃたべました。

「ねえお母さん。ぼくお父さんはきっとまもなく帰ってくると思うよ」

「ああ、あたしもそう思う。けれどもおまえはどうしてそう思うの」

「だって\ruby[g]{今朝}{けさ}の新聞に今年は北の方の\ruby{漁}{りょう}はたいへんよかったと書いてあったよ」

「ああだけどねえ、お父さんは\ruby{漁}{りょう}へ出ていないかもしれない」

「きっと出ているよ。お父さんが\ruby{監獄}{かん|ごく}へはいるようなそんな\ruby{悪}{わる}いことをしたはずがないんだ。この前お父さんが持ってきて学校へ\ruby{寄贈}{き|ぞう}した\ruby{巨}{おお}きな\ruby{蟹}{かに}の\ruby{甲}{こう}らだのとなかいの\ruby{角}{つの}だの今だってみんな\ruby{標本室}{ひょう|ほん|しつ}にあるんだ。六年生なんか\ruby{授業}{じゅ|ぎょう}のとき先生がかわるがわる教室へ\ruby{持}{も}って行くよ」

「お父さんはこの\ruby{次}{つぎ}はおまえにラッコの\ruby{上着}{うわ|ぎ}をもってくるといったねえ」

「みんながぼくにあうとそれを\ruby{言}{い}うよ。ひやかすように\ruby{言}{い}うんだ」

「おまえに\ruby{悪口}{わる|くち}を\ruby{言}{い}うの」

「うん、けれどもカムパネルラなんか\ruby{決}{けっ}して\ruby{言}{い}わない。カムパネルラはみんながそんなことを\ruby{言}{い}うときはきのどくそうにしているよ」

「カムパネルラのお父さんとうちのお父さんとは、ちょうどおまえたちのように小さいときからのお\ruby{友達}{とも|だち}だったそうだよ」

「ああだからお父さんはぼくをつれてカムパネルラのうちへもつれて行ったよ。あのころはよかったなあ。ぼくは学校から帰る\ruby{途中}{と|ちゅう}たびたびカムパネルラのうちに\ruby{寄}{よ}った。カムパネルラのうちにはアルコールランプで走る汽車があったんだ。レールを七つ組み合わせるとまるくなってそれに\ruby{電柱}{でん|ちゅう}や\ruby{信号標}{しん|ごう|ひょう}もついていて\ruby{信号標}{しん|ごう|ひょう}のあかりは汽車が通るときだけ青くなるようになっていたんだ。いつかアルコールがなくなったとき\ruby{石油}{せき|ゆ}をつかったら、\ruby{缶}{かん}がすっかりすすけたよ」

「そうかねえ」

「いまも毎朝新聞をまわしに行くよ。けれどもいつでも家じゅうまだしいんとしているからな」

「早いからねえ」

「ザウエルという犬がいるよ。しっぽがまるで\ruby{箒}{ほうき}のようだ。ぼくが行くと\ruby{鼻}{はな}を鳴らしてついてくるよ。ずうっと町の\ruby{角}{かど}までついてくる。もっとついてくることもあるよ。今夜はみんなで\ruby{烏瓜}{からす|うり}のあかりを川へながしに行くんだって。きっと犬もついて行くよ」

「そうだ。\ruby{今晩}{こん|ばん}は\ruby{銀河}{ぎん|が}のお\ruby{祭}{まつ}りだねえ」

「うん。ぼく\ruby{牛乳}{ぎゅう|にゅう}をとりながら見てくるよ」

「ああ行っておいで。川へははいらないでね」

「ああぼく\ruby{岸}{きし}から見るだけなんだ。一時間で行ってくるよ」

「もっと\ruby{遊}{あそ}んでおいで。カムパネルラさんといっしょなら\ruby{心配}{しん|ぱい}はないから」

「ああきっといっしょだよ。お母さん、窓をしめておこうか」

「ああ、どうか。もう\ruby{涼}{すず}しいからね」

ジョバンニは立って\ruby{窓}{まど}をしめ、お\ruby{皿}{さら}やパンの\ruby{袋}{ふくろ}をかたづけると\ruby{勢}{いきお}いよく\ruby{靴}{くつ}をはいて、

「では一時間\ruby{半}{はん}で帰ってくるよ」と\ruby{言}{い}いながら\ruby{暗}{くら}い\ruby{戸口}{と|ぐち}を出ました。

